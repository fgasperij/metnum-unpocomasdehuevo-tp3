\documentclass[11pt, a4paper]{article}
\usepackage{algorithmic}
%\usepackage{savesym}
%\savesymbol{IF, algorithmic}
\usepackage[utf8]{inputenc}
\usepackage[spanish]{babel}
\usepackage[paper=a4paper, left=2.4cm, right=2cm, bottom=2cm, top=2.4cm]{geometry}
\usepackage{aed2-symb}
\usepackage{textcomp}
\usepackage[section, ruled]{algorithm}
\usepackage{hyperref}
\usepackage[pdftex]{graphicx}
\usepackage{epsfig}

%\floatname{algorithm}{Algoritmo} % para que diga ``Algoritmo'' y no ``Algorithm''

\renewcommand{\algorithmiccomment}[1]{\ \ \ \textbf{//#1}} %para agregar comntarios en los algoritmos poner
% \COMMENT{ALAN ES GAY}
% \usepackage{algpseudocode}
% \usepackage{lineno}
\newenvironment{parr}
{\begin{list}{}%
         {\setlength{\leftmargin}{2em}}%
         \item[]%
}
{\end{list}}
\newcommand{\tab}{\hspace*{2em}}

\newcommand{\oper}[3]{\samepage\textsc{#1}(#2) $\rightarrow$ \textit{res} $\colon$ #3\\*}
\newcommand{\operL}[3]{\samepage\textsc{#1}(#2) $\rightarrow$ \textit{res} $\colon$ #3}
\newcommand{\operM}[2]{\samepage\textsc{#1}(#2)\\} %ésta es la operación que no devuelve nada
\newcommand{\operMA}[2]{\samepage\textsc{#1}(#2)} %ésta es la operación que no devuelve nada y para usar en el
%``caption'' del ``algorithm''
\newcommand{\vin}[2]{\textbf{in} \textit{#1}$\colon$#2}
\newcommand{\vinout}[2]{\textbf{in/out} \textit{#1}$\colon$#2}
\newcommand{\pre}[1]{\textbf{Pre} $\equiv$ \{#1\}\\*}
\newcommand{\post}[1]{\textbf{Post} $\equiv$ \{#1\}\\*}
\newcommand{\complej}[1]{\textbf{Complejidad:} #1\\*}
\newcommand{\desc}[1]{\textbf{Descripci\'on:} #1\\}
\newcommand{\alias}[1]{\textbf{Aliasing:} #1\\}

% \oper{nombre}{recibe}{devuelve}
% \vin{var}{tipo}
% \vinout{var}{tipo}
% \pre{}
% \post{}
% \complej{O()}
% \desc{txt}
% \alias{txt}

\newcommand{\eqobs}{$=_{obs}$}
\newcommand{\bigo}[1]{O$\big($#1$\big)$}
\newcommand{\NULL}{\textsc{Null} }
\newcommand{\aux}[3]{\samepage #1$\colon$ #2 $\rightarrow$ #3\\*}

\usepackage{metnum}
   
%Datos para la caratula
\materia{Métodos Numéricos}
\titulo{TP 3 - Un poco más de huevo...}
\grupo{Grupo 2}
\integrante{Franco Bartalotta}{347/11}{franco.bartalotta@hotmail.com}
\integrante{Fernando Gasperi Jabalera}{56/09}{fgasperijabalera@gmail.com}
\integrante{Erik Machicado}{067/12}{erik.machicado@hotmail.com}
\integrante{Ana Sarri\'es}{144/02}{abarloventos@gmail.com}

\resumen{El presente trabajo tiene como objetivo, predecir con la mayor anticipación posible 
la posición futura de una pelota en el campo de fútbol, utilizando la posición de la 
misma en el pasado, para coordinar los movimientos de nuestro arquero.
Para resolver este problema computacionalmente, se decidió considerar el tiempo de manera discreta, 
el campo de futbol como un plano de $\mathbb{R}^2$.}
\claves{Cuadrados Mínimos, Arquero robot.}

\begin{document}  
\maketitle
\tableofcontents

\newpage
\section{Introducci\'on Te\'orica}


\section{Desarrollo}

\subsection{Descripción del método}
Dado un instante $t$ nosotros contamos con $(t+1)$ mediciones de la posición de la pelota que se corresponden con los intantes 
0, 1, ..., $t-1$, $t$. 
\begin{center}
  \begin{tabular}{l | l*{3}{c}r}
  t              & 0 & 1 & ... & (t-1) & t\\
  \hline
  x(t)	       & x(0) & x(1) & ... & x(t-1) & x(t)\\
  \hline
  y(t)	       & y(0) & y(1) & ... & y(t-1) & y(t)\\
  \end{tabular} 
\end{center}
De esta forma la posición de la pelota queda descripta por una función $p:\mathbb{R} \to \mathbb{R}^2$ tal que $p(t) = (x(t),y(t))$.
Nuestro procedimiento para decidir el sentido y la magnitud del desplazamiento del arquero para un instante $t$ consiste de los siguientes
pasos:
\begin{enumerate}
 \item aproximamos la función $x(t)$ con cuadrados mínimos obteniendo $P_x(t)$
 \item calculamos una raíz de $Q_x(t) = P_x(t) - x_{posicionDelArco}$ para conocer en qué instante $t_{raiz}$ atraviesa la línea de meta
 \item aproximamos la función $y(t)$ con cuadrados mínimos obteniendo $P_y(t)$
 \item evaluamos $P_y(t)$ en $t_{raiz}$ para obtener $P_y(t_{raiz}) = y_{aproximacion}$
 \item calculamos el desplazamiento óptimo del arquero asumiendo que el disparo ingresará por $y_{aproximacion}$
\end{enumerate}

El planteo formal del método de cuadrados mínimos para la aproximación de la función $x(t)$ (para $y(t)$ es análogo) es el siguiente:
\begin{displaymath}
\begin{pmatrix}
  0^n & \cdots    & 0^1 & 1	\\
  \vdots &  \vdots &  \vdots & \vdots  \\ 
  \vdots &  \vdots &  \vdots & \vdots  \\ 
  t^n & \cdots & t^1 & 1\\
\end{pmatrix}
*
\begin{pmatrix}
 a_n\\
 \vdots\\
 a_0\\
\end{pmatrix}
 =
\begin{pmatrix}
  x(0)\\
  \vdots \\
  \vdots \\
  x(t)\\ 
\end{pmatrix} 
\end{displaymath}

donde la cantidad de filas se corresponde con la cantidad de instantes $t$ que consideramos de la trayectoria y $n$ es el grado del 
polinomio con el que estimamos a $x(t)$. Resolviendo el sistema de ecuaciones normales correspondiente al sistema anterior, obtenemos los coeficientes de $P_x(t)$. Para resolverlo utilizamos eliminación gaussiana por ser un métodos de simple
implementación. Siempre podemos aplicar eliminación gaussiana porque el grado del polinomio que utilizamos para aproximar por cuadrados mínimos, $g$,  es como máximo menor o igual a la cantidad de mediciones que tenemos menos 2. Esto lo decidimos así para nunca utilizar un polinomio interpolador, ya que el mismo sería sensible hasta a los errores de medición.
\begin{displaymath}
	g \leq \#mediciones - 2 \Longrightarrow \#columnas(A) \leq \#filas(A) - 2
\end{displaymath}
por lo tanto la matriz $A$ va a tener rango columna completo.
 La solución son los $a_i$ con los que construimos el polinomio que aproxima a $x(t)$ de la siguiente forma:
\begin{displaymath}
	P_x(t)=a_n*t^n + \cdots + a_1*t + a_0 
\end{displaymath}
Así resulta que $P_x(t)$ es el polinomio de grado $n$ que minimiza la suma del error cuadrático con respecto a las mediciones. 
\par
La línea de meta se encuentra en un valor fijo de $x$ al que nos referiremos como $x_{posicionDelArco}$. El arquero siempre está 
situado sobre ese valor $x_{posicionDelArco}$ y se mueve a lo largo del eje $y$. Nuestra estimación debe ser el valor de $y$ por el
cual el balón atravesará la línea de meta. Si $p(t) = (x(t), y(t))$ describe la trayectoria del balón y asumimos que existe un $t_{gol}$ para el 
cual:
\begin{displaymath}
  p(t_{gol}) = (x(t_{gol}), y(t_{gol})) = (x_{posicionDelArco}, y(t_{gol}))
\end{displaymath}
a nosotros nos interesa aproximar $y(t)$ para poder evaluarla en $t_{gol}$ y así conocer en qué valor de $y$ el balón atravesará la línea
de meta. Para aproximar $y(t)$ utilizamos el método de cuadrados mínimos ya explicado para aproximar $x(t)$ y obtenemos un $P_y(t)$. Luego,
vamos a calcular $t_{gol}$ con nuestro $P_x(t) \approx x(t)$:
\begin{align}
    P_x(t) = x_{posicionDelArco}  \Longleftrightarrow t = t_{gol}\\
    P_x(t) - x_{posicionDelArco} = 0 \Longleftrightarrow t = t_{gol}
\end{align}
Tomando $Q_x(t) = P_x(t) - x_{posicionDelArco}$ sabemos que si tiene raíces reales mayores al último instante $t$ de la trayectoria 
entonces $t_{gol}$ es una de ellas. Para encontrarla utilizamos el método de bisección \cite[]{BoostSite}. 
El intervalo inicial $[a,b]$ siempre es el mismo con $a = 0$ y $b = 10000$. Sabemos que $Q_x(0) > 0$ porque el disparo siempre se origina
dentro del campo de juego. Necesitamos un valor $b$ para el cual sepamos que $Q_x(b) < 0$. Esto es lo mismo que pedir un instante $t$
para el cual la posición de la pelota se encuentre del otro lado de la línea de meta según nuestra aproximación por cuadrados mínimos. 
Tomamos un valor lo suficientemente grande como para asegurarnos que el disparo haya terminado antes del mismo. Es cierto que esto
no nos asegura que $Q_x(b) < 0$ porque podría ser que $Q_x(t)$ tuviera más de una raíz real mayor a 0. Si nos encontraramos con este caso
lo descartaríamos. Sin embargo, consideramos que las probabilidades son bajas ya que para que el disparo termine en gol debe ir acercándose
a la línea de meta hasta finalmente cruzarla. Que se vaya acercando hasta finalmente cruzarla lo traducimos a que $Q_x(t)$ debe decrecer hasta finalmente
cruzar el eje x. Que $Q_x(t)$ vuelva a crecer y así tener otra raíz real intuimos que no es probable dado que no hay mediciones que se lo propongan,
las últimas deben hacerla decrecer hasta cruzar la línea de meta. 
Una vez que obtenemos 
$t_{gol}$ evaluamos: 
\begin{displaymath}
  P_y(t_{gol}) = y_{aproximacion}
\end{displaymath}
$y_{aproximacion}$ será la posición que el arquero intentará cubrir.
\par

\subsection{Elección de las aproximaciones}
Un caso particular es cuando contamos con sólo una sólo medición de la posición del balón ya que no podemos aplicar cuadrados mínimos. En
ese caso aproximamos por la recta constante. Es decir, que cuando nuestra única medición es $p(0) = (x, y)$ nuestra estimación es que la 
pelota atravesará la línea de meta en $y$. Esta decisión no tiene ningún soporte matemático ni experimental, simplemente nos resultó
intuitiva.
\par
Como las mediciones de la posición del balón pueden tener errores propios de los instrumentos utilizados o ruidos de fuentes circunstanciales
evitamos aproximar con el polinomio interpolador ya que hacerlo generaría que el mismo, $P_{x/y}(t)$, sea sensible a estos errores. Es por esta
razón que cuadrados mínimos, con un grado menor al interpolador, resulta en una aproximación más fiable de la función original aunque 
no atraviese todos los puntos obtenidos por las mediciones. Dados $n$ puntos $(x_i, y_i)$ sabemos que siempre existe un polinomio único 
que los interpola de grado menor o igual a $(n-1)$. Por lo tanto, como en el instante $t$ contamos con una trayectoria de $t+1$
posiciones nuestro $P_{x/y}(t)$ siempre lo construimos con un grado menor a $t$.
\par
Cuando intentamos calcular la raíz de $Q_x(t)$ que se corresponde con $t_{gol}$ y no lo conseguimos simplemente descartamos ese $P_x(t)$
e intentamos con otro de grado diferente. Sin embargo, un polinomio tiene un número de raíces igual a su grado por lo tanto cuanto más 
alto el grado mayor la cantidad de raíces y esto redunda en una mayor dificultad para encontrar la raíz correspondiente a $t_{gol}$. Para
intentar compensar este efecto sólo tomamos las últimas mediciones que efectivamente se acercan a la línea de meta. Si nuestras mediciones
se corresponden con los puntos $(x_0, y_0)$, $(x_1, y_1)$, ..., $(x_t, y_t)$ entonces sólo consideraremos los últimos $p_i$ puntos tales que:
\begin{displaymath}
x_i > x_{i-1}
\end{displaymath}
es decir, los últimos $p_i$ que efectivamente se están acercando a la meta.

\subsection{Errores de medición}\label{ssec:errores_de_medicion}
En general asumimos que los errores de medición son pequeños, sin embargo, puede ocurrir que debido a una circunstancia extraordinaria
se registren algunos valores con un error muy grande. Estos outliers deben reconocerse y tratarse de un modo particular para que perturben en
la menor medida posible nuestras aproximaciones. Nuestro 
tratamiento de los mismos será ignorarlos, es decir, continuar aproximando como si esa medición no hubiera sido realizada.  En cada instante $t$ 
aproximamos el $y_gol$ para mover al arquero y además calculamos: 
\begin{displaymath}
	p_{t+1} = (x_{t+1}, y_{t+1})
\end{displaymath}
En el instante $t+1$ comparamos la medición que recibimos con nuestra previa aproximación $p_{t+1}$. Calculamos el error relativo total de la siguiente
forma:
\begin{displaymath}
	e_{relativoX} = \frac{|xAproximacion_{t+1} - xMedicion_{t+1}|}{xMedicion_{t}}
\end{displaymath}
\begin{displaymath}
	e_{relativoTotal} = \frac{e_{relativoX} + e_{relativoY}}{2}
\end{displaymath}
Si el error relativo total resulta ser mayor al $80\%$ entonces lo que hacemos es descartar esa medición y utilizar nuestra previa aproximación. Luego,
en el instante $t+2$ utilizaremos como medición del instante $t+1$ una interpolación cuadrática entre los valores de la medición del instante $t$ y 
la del instante $t+2$. Es necesario aclarar que la detección de outliers sólo se considera válida si evaluamos que el cambio no puede deberse a que un jugador rival
alteró la trayectoria del balón. Consideramos que un jugador puede haber alterado la trayectoria del balón si el mismo pasa a una distancia menor a 7, la misma
que se utiliza para considerar si el arquero atajó o no el disparo.

\subsection{Jugadores rivales}
La posición de los jugadores rivales juega un papel muy importante al momento de aproximar $p(t)$ ya que un jugador puede 
patear la pelota y cambiar por completo su dirección y su velocidad. En primer lugar, si el balón pasa a una distancia menor a un umbral
determinado de antemano de la posición de un jugador a la próxima medición no le aplicamos la detección de errores de medición
extraordinarios que explicamos en \ref{ssec:errores_de_medicion}. Además una vez que el balón pasa por un jugador vamos a 
dejar de considerar las mediciones anteriores ya que si el jugador patéo el balón es efectivamente un disparo nuevo y por lo tanto las
mediciones anteriores no aportan información alguna.


















\section{Experimentación}

\subsection{Discusión sobre el método y sus variables}
La efectividad del método estará dada por la calidad de la aproximación del lugar por el cual el disparo llegará a la línea de meta. 
Nuestra aproximación de $x(t)$ nos da el instante en el cual el balón atravesará la línea de meta y luego nuestra aproximación
de $y(t)$ nos dirá el lugar por dónde lo hará. Las variables principales que entran en juego al aproximar una función con 
cuadrados mínimos son las mediciones a considerar, es decir los puntos, y el grado del polinomio con el cual se aproximará la función.
Creemos que un análisis físico sobre las diferentes fuerzas que puedan afectar a la dirección y aceleración del balón puede indicar 
más firmemente qué grados tiene más sentido utilizar para cuadrados mínimos. Sin embargo, el mismo cae fuera del alcance del presente 
trabajo por lo cual nos limitaremos a experimentar con el mayor rango de grados posible y quizás los resultados nos arrojen alguna pista.
Con respecto a qué puntos de la trayectoria utilizar intuimos que no todos los puntos de la trayectoria proveen información de la misma 
calidad. Es por eso que destinaremos un experimento entero a evaluar qué tasa de efectividad obtenemos tomando diferentes subconjuntos 
de puntos de la trayectoria.

\subsection{Bases generales}
Criterios que aplicamos a todos los experimentos:
\begin{enumerate}
	\item Para determinar si cierta combinación de parámetros tuvo una mejor performance que otra, vamos a basarnos exclusivamente en la tasa 
	de efectividad de atajadas del arquero. Solamente vamos a considerar si nuestra aproximación converge a la posición final en el experimento para 
	evaluar el método.
	\item Para generar instancias basadas en una función polinómica de grado $g$ tomamos un punto al azar del campo de juego y otro al azar entre los dos postes 
del arco. Luego tomamos tantos puntos al azar sobre el campo de juego entre el primer punto y la línea de meta hasta alcanzar una cantidad
de puntos igual a $\#p = g + 1$. Con esos puntos generamos el polinomio de Lagrange asociado y con el mismo generamos el resto de los puntos
que queramos para nuestra trayectoria.
	\item 
\end{enumerate}

\subsection{Experimentos}

\subsubsection{Evaluación del método}
El objetivo de esta experimentación es evaluar si la aproximación de nuestro método realmente converge al valor final de 
la posición $y$ por la cual el balón atraviesa la línea de meta. Lo haremos con lo que consideramos disparos básicos de una situación
real de juego. Éstos comprenden disparos con trayectorias de tres tipos: rectas, parábolas y curvas generadas por polinomios de tercer
grado. Separaremos en esos tres tipos y generaremos para cada uno de ellos 100 instancias aleatorias. Todas ellas compartirán 
una cantidad de mediciones igual a 30 y el gráfico que presentaremos se constituirá por los promedios de las instancias de cada tipo. 
Para obtener la información pertinente iremos calculando en cada instante la distancia entre 
el $y_{gol}$ real y nuestra aproximación del mismo. De esta forma podremos ver si nuestra aproximación converge al valor final de 
la posición en la que el balón atraviesa la línea de meta.
Además introduciremos un ruido de hasta $5\%$, es decir, el valor de la medición generada puede variar hasta en un $5\%$ con 
respecto al valor original de la función en el punto. La forma en la que introducimos el ruido es la siguiente:
\begin{enumerate}
	\item cada punto tiene una probabilidad del $50\%$ de llevar ruido o no
	\item el porcentaje de ruido introducido es un valor del intervalo $[0, 5]$ elegido aleatoriamente con una distribución uniforme
	\item el ruido será sumado o restado también con una probabilidad del $50\%$
\end{enumerate}
Dejaremos de lado a los jugadores y los errores extraordinarios en este experimento ya que nos interesa simplemente evaluar cómo se 
comporta el método con trayectorias simples.  

\subsubsection{Grado del polinomio}
El objetivo de este experimento es analizar si la tasa de efectividad mejora aproximando las funciones $x(t)$ e $y(t)$ con un grado en 
particular de polinomio de cuadrados mínimos. Probaremos desde el primer grado hasta el décimo grado. La idea principal detrás de esta 
experimentación es calcular cual es el grado óptimo para que la tasa de efectividad sea lo mejor posible. Se presume que un grado muy 
bajo no va a ser eficiente ya que no puede interpolar muchas puntos y va a ser malo para aproximar curvas algo complicadas, y tampoco 
lo va a ser un grado alto, ya que suelen ser curvas muy oscilantes que pueden introducir error. Para cada grado se testearon todas las instancias 
fijando la cantidad de puntos a considerar (siempre los últimos) y se tomo el promedio de dichos resultados. Se toman instancias de todo tipo, 
tanto de la cátedra como propias.

\subsubsection{Puntos a considerar}
Para esta experimentación se desea ver cómo afecta la cantidad de puntos que se consideran para hallar el polinomio de aproximación de 
cuadrados mínimos. Siempre se van a considerar los últimos puntos medidos. En principio, una cantidad baja de puntos podría no ser 
buena ya que no son suficientes para estimar curvas complicadas, tampoco sería bueno que la cantidad sea demasiado alta porque la 
estimación podría ser más propensa a errores, especialmente considerando la presencia de jugadores que pueden alterar el curso de la 
pelota de manera inesperada y que no corresponda con alguna curva "natural''. Como en la experimentación anterior, se usan todo tipo 
de instancias.


%\subsubsection{Errores extraordinarios}

\newpage

% Bibliografía
\addcontentsline{toc}{section}{Referencias}
\bibliography{tp3.bib}{}
\bibliographystyle{acm}
\end{document}
