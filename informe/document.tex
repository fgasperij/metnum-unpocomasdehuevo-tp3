\documentclass[11pt, a4paper]{article}
\input{macrostp2.tex}
\usepackage{metnum}
   
%Datos para la caratula
\materia{Métodos Numéricos}
\titulo{TP 3 - Un poco más de huevo...}
\grupo{Grupo 2}
\integrante{Franco Bartalotta}{347/11}{franco.bartalotta@hotmail.com}
\integrante{Fernando Gasperi Jabalera}{56/09}{fgasperijabalera@gmail.com}
\integrante{Erik Machicado}{067/12}{erik.machicado@hotmail.com}
\integrante{Ana Sarri\'es}{144/02}{abarloventos@gmail.com}

\resumen{El presente trabajo tiene como objetivo, predecir con la mayor anticipación posible 
la posición futura de una pelota en el campo de fútbol, utilizando la posición de la 
misma en el pasado, para coordinar los movimientos de nuestro arquero.
Para resolver este problema computacionalmente, se decidió considerar el tiempo de manera discreta, 
el campo de futbol como un plano de $\mathbb{R}^2$.}
\claves{Cuadrados Mínimos Lineales, Arquero robot}

\begin{document}  
\maketitle
\tableofcontents

\newpage

\section{Introducción}

Para la predicción de la posición por donde entrará la pelota, nos basamos en el método de \emph{cuadrados mínimos lineales}. Este método es particularmente adecuado para los casos en que se introducen errores de medición\cite[\emph{3.2}]{heath} como ocurre en las mediciones de la trayectoria de nuestro tp (según se advierte en el enunciado).
\par
Recordemos que la tratectoria de la pelota se puede formalizar mediante la función $p:\mathbb{R} \to \mathbb{R}^2$, $p(t) = (x(t),y(t))$ donde t tomará valores de tiempo discretizado. Nuestra estrategia consiste en calcular los coeficientes de dos polinomios $P_x(t)$ y $P_y(t)$ que aproximen a las funciones $x(t)$ e $y(t)$, respectivamente. 

\par
El planteo formal del método de CM para $P_x(t)$ (para $P_y(t)$ es análogo) es el siguiente:
\par
$
\begin{pmatrix}
  0^G & \cdots    & 0^1 & 1	\\
  \vdots &  \vdots &  \vdots & \vdots  \\ 
  \vdots &  \vdots &  \vdots & \vdots  \\ 
  T^G & \cdots & T^1 & 1\\
\end{pmatrix}
*
\begin{pmatrix}
 coefG\\
 \vdots\\
 coef0\\
\end{pmatrix}
 =
\begin{pmatrix}
  x(0)\\
  \vdots \\
  \vdots \\
  x(T)\\ 

\end{pmatrix}
$
\par
donde la cantidad de filas se corresponde con la cantidad de instantes \emph{T} que consideramos de la trayectoria de un tiro en particular (comenzando desde el instante cero) y donde \emph{G} es el grado del polinomio con el que estimamos a $x(t)$ (lo elegimos siempre menor a la cantidad de instantes considerados).
\par
Así el polinomio que aproxima a $x(t)$ sería $P_x(t)=coefG*t^G + \cdots + coef1*t + coef0$
\par
Como en nuestro modelo, la pelota se atajará sobre la recta x=125, calcularemos entonces la raiz $t_{x=125}$ del polinomio $\hat{P}_x(t) = P_x(t) -125$; de manera que $P_y(t_{x=125})$ será la posición que nuestro arquero intentará cubrir en caso de encontrarse entre los límites del arco.

\section{Desarrollo}

\subsection{Descripción del método}
Dado un instante $t$ nosotros contamos con $(t+1)$ mediciones de la posición de la pelota que se corresponden con los intantes 
0, 1, ..., $t-1$, $t$. 
\begin{center}
  \begin{tabular}{l | l*{3}{c}r}
  t              & 0 & 1 & ... & (t-1) & t\\
  \hline
  x(t)	       & x(0) & x(1) & ... & x(t-1) & x(t)\\
  \hline
  y(t)	       & y(0) & y(1) & ... & y(t-1) & y(t)\\
  \end{tabular} 
\end{center}
De esta forma la posición de la pelota queda descripta por una función $p:\mathbb{R} \to \mathbb{R}^2$ tal que $p(t) = (x(t),y(t))$.
Nuestro procedimiento para decidir el sentido y la magnitud del desplazamiento del arquero para un instante $t$ consiste de los siguientes
pasos:
\begin{enumerate}
 \item aproximamos la función $x(t)$ con cuadrados mínimos y obtenemos $P_x(t)$
 \item calculamos una raíz de $x'(t) = x(t) - x_{posicionDelArco}$ para conocer en qué instante $t_{raiz}$ atraviesa la línea de meta
 \item aproximamos la función $y(t)$ con cuadrados mínimos y obtenemos $P_y(t)$
 \item evaluamos $P_y(t)$ en $t_{raiz}$ para obtener $P_y(t_{raiz}) = y_{aproximacion}$
 \item calculamos el desplazamiento óptimo del arquero asumiendo que el disparo ingresará por $y_{aproximacion}$
\end{enumerate}

La aproximación de la función $x(t)$ (para $y(t)$ es análogo) mediante el método de cuadrados mínimos la hacemos resolviendo el sistema:
\begin{displaymath}
\begin{pmatrix}
  0^n & \cdots    & 0^1 & 1	\\
  \vdots &  \vdots &  \vdots & \vdots  \\ 
  \vdots &  \vdots &  \vdots & \vdots  \\ 
  t^n & \cdots & t^1 & 1\\
\end{pmatrix}
*
\begin{pmatrix}
 a_n\\
 \vdots\\
 a_0\\
\end{pmatrix}
 =
\begin{pmatrix}
  x(0)\\
  \vdots \\
  \vdots \\
  x(t)\\ 
\end{pmatrix} 
\end{displaymath}

donde la cantidad de filas se corresponde con la cantidad de instantes $t$ que consideramos de la trayectoria y $n$ es el grado del 
polinomio con el que estimamos a $x(t)$. Para resolver el sistema utilizamos eliminación gaussiana por ser un métodos de simple
implementación. Una vez resuelto el sistema utilizamos los $a_i$ para construir el polinomio que aproxima a $x(t)$ de la siguiente forma:
\begin{displaymath}
P_x(t)=a_n*t^n + \cdots + a_1*t + a_0 
\end{displaymath}
Sabemos que $P_x(t)$ es el polinomio de grado $n$ que minimiza la suma del error cuadrático con respecto a las mediciones. 
\par
La línea de meta se encuentra en un valor fijo de $x$ al que nos referiremos como $x_{posicionDelArco}$. El arquero siempre está 
situado sobre ese valor $x_{posicionDelArco}$ y se mueve a lo largo del eje $y$. Nuestra estimación debe ser el valor de $y$ por el
cual el balón atravesará la línea de meta. Si $p(t) = (x(t), y(t))$ describe la trayectoria del balón y asumimos que existe un $t_{gol}$ para el 
cual:
\begin{displaymath}
  p(t_{gol}) = (x(t_{gol}), y(t_{gol})) = (x_{posicionDelArco}, y(t_{gol}))
\end{displaymath}
a nosotros nos interesa aproximar $y(t)$ para poder evaluarla en $t_{gol}$ y así conocer en qué valor de $y$ el balón atravesará la línea
de meta. Para aproximar $y(t)$ utilizamos el método de cuadrados mínimos ya explicado para aproximar $x(t)$ y obtenemos un $P_y(t)$. Luego,
vamos a calcular $t_{gol}$ con nuestro $P_x(t) \approx x(t)$:
\begin{align}
    P_x(t) = x_{posicionDelArco}  \Longleftrightarrow t = t_{gol}\\
    P_x(t) - x_{posicionDelArco} = 0 \Longleftrightarrow t = t_{gol}\\
\end{align}
Tomando $P'_x(t) = P_x(t) - x_{posicionDelArco}$ sabemos que si tiene raíces reales mayores al último instante $t$ de la trayectoria 
entonces $t_{gol}$ es una de ellas. Para encontrarla utilizamos el método de bisección \cite[]{BoostSite}. Una vez que obtenemos 
$t_{gol}$ evaluamos: 
\begin{displaymath}
  P_y(t_{gol}) = y_{aproximacion}
\end{displaymath}
$y_{aproximacion}$ será la posición que el arquero intentará cubrir.
\par

\subsection{Elección de las aproximaciones}
Un caso particular es cuando contamos con sólo una sólo medición de la posición del balón ya que no podemos aplicar cuadrados mínimos. En
ese caso aproximamos por la recta constante. Es decir, que cuando nuestra única medición es $p(0) = (x, y)$ nuestra estimación es que la 
pelota atravesará la línea de meta en $y$. Esta decisión no tiene ningún soporte matemático ni experimental, simplemente nos resultó
intuitiva.
\par
Como las mediciones de la posición del balón pueden tener errores propios de los instrumentos utilizados o ruidos de fuentes circunstanciales
evitamos aproximar con el polinomio interpolador ya que hacerlo generaría que el mismo, $P_{x/y}(t)$, sea sensible a estos errores. Es por esta
razón que cuadrados mínimos, con un grado menor al interpolador, resulta en una aproximación más fiable de la función original aunque 
no atraviese todos los puntos obtenidos por las mediciones. Dados $n$ puntos $(x_i, y_i)$ sabemos que siempre existe un polinomio único 
que los interpola de grado menor o igual a $(n-1)$. Por lo tanto, como en el instante $t$ contamos con una trayectoria de $t+1$
posiciones nuestro $P_{x/y}(t)$ siempre lo construimos con un grado menor a $t$.
\par
Cuando intentamos calcular la raíz de $P'_x(t)$ que se corresponde con $t_{gol}$ y no lo conseguimos simplemente descartamos ese $P_x(t)$
e intentamos con otro de grado diferente. Sin embargo, un polinomio tiene un número de raíces igual a su grado por lo tanto cuanto más 
alto el grado mayor la cantidad de raíces y esto redunda en una mayor dificultad para encontrar la raíz correspondiente a $t_{gol}$. Para
intentar compensar este efecto sólo tomamos las últimas mediciones que efectivamente se acercan a la línea de meta. Si nuestras mediciones
se corresponden con los puntos $(x_0, y_0)$, $(x_1, y_1)$, ..., $(x_t, y_t)$ entonces sólo consideraremos los puntos $p_i$ tales que:
\begin{displaymath}
 (\forall j \in \mathbb{N}) j > i \Rightarrow x_j < x_{j-1}
\end{displaymath}

\subsection{Errores de medición}\label{ssec:errores_de_medicion}
En general asumimos que los errores de medición son pequeños, sin embargo, puede ocurrir que debido a una circunstancia extraordinaria
se registren algunos valores con un error muy grande. Estos outliers deben reconocerse y tratarse de un modo particular para que perturben en
la menor medida posible nuestras aproximaciones. En la sección de experimentación se probará un método para reconocerlos y nuestro 
tratamiento de los mismos será ignorarlos, es decir, continuar aproximando como si esa medición no hubiera sido realizada. Si esa medición
fue la correspondiente a el instante $t$, cuando obtengamos la medición del instante $t+1$ lo que haremos será extrapolar cuadráticamente
la medición del instante $t$ para no sumarle velocidad innecesaria a nuestras aproximaciones.

\subsection{Jugadores rivales}
La posición de los jugadores rivales juega un papel muy importante al momento de aproximar $p(t)$ ya que un jugador puede 
patear la pelota y cambiar por completo su dirección y su velocidad. En primer lugar, si el balón pasa a una distancia menor a un umbral
determinado de antemano de la posición de un jugador a la próxima medición no le aplicamos la detección de errores de medición
extraordinarios que explicamos en \ref{ssec:errores_de_medicion}. Además una vez que el balón pasa por un jugador vamos a 
dejar de considerar las mediciones anteriores ya que si el jugador patéo el balón es efectivamente un disparo nuevo y por lo tanto las
mediciones anteriores no aportan información alguna.


















\section{Experimentación}

\subsection{Discusión sobre el método y sus variables}
La efectividad del método estará dada por la calidad de la aproximación del lugar por el cual el disparo llegará a la línea de meta. 
Nuestra aproximación de $x(t)$ nos da el instante en el cual el balón atravesará la línea de meta y luego nuestra aproximación
de $y(t)$ nos dirá el lugar por dónde lo hará. Las variables principales que entran en juego al aproximar una función con 
cuadrados mínimos son las mediciones a considerar, es decir los puntos, y el grado del polinomio con el cual se aproximará la función.
Creemos que un análisis físico sobre las diferentes fuerzas que puedan afectar a la dirección y aceleración del balón puede indicar 
más firmemente qué grados tiene más sentido utilizar para cuadrados mínimos. Sin embargo, el mismo cae fuera del alcance del presente 
trabajo por lo cual nos limitaremos a experimentar con el mayor rango de grados posible y quizás los resultados nos arrojen alguna pista.
Con respecto a qué puntos de la trayectoria utilizar intuimos que no todos los puntos de la trayectoria proveen información de la misma 
calidad. Es por eso que destinaremos un experimento entero a evaluar qué tasa de efectividad obtenemos tomando diferentes subconjuntos 
de puntos de la trayectoria.

\subsection{Bases generales}
Criterios que aplicamos a todos los experimentos:
\begin{itemize}
	\item Para determinar si cierta combinación de parámetros tuvo una mejor performance que otra, vamos a basarnos exclusivamente en la tasa 
	de efectividad de atajadas del arquero. Solamente vamos a considerar si nuestra aproximación converge a la posición final en el experimento para 
	evaluar el método.
	\item Para generar instancias basadas en una función polinómica de grado $g$ tomamos un punto al azar del campo de juego y otro al azar entre los dos postes 
del arco. Luego tomamos tantos puntos al azar sobre el campo de juego entre el primer punto y la línea de meta hasta alcanzar una cantidad
de puntos igual a $\#p = g + 1$. Con esos puntos generamos el polinomio de Lagrange asociado y con el mismo generamos el resto de los puntos
que queramos para nuestra trayectoria.
	\item El grado del polinomio utilizado en los experimentos no est
	\begin{enumerate}
		\item
	\end{enumerate}
\end{itemize}

\subsection{Experimentos}

\subsubsection{Evaluación del método}
El objetivo de esta experimentación es evaluar si la aproximación de nuestro método realmente converge al valor final de 
la posición $y$ por la cual el balón atraviesa la línea de meta. Lo haremos con lo que consideramos disparos básicos de una situación
real de juego. Éstos comprenden disparos con trayectorias de tres tipos: rectas, parábolas y curvas generadas por polinomios de tercer
grado. Separaremos en esos tres tipos y generaremos para cada uno de ellos 100 instancias aleatorias. Todas ellas compartirán 
una cantidad de mediciones igual a 30 y el gráfico que presentaremos se constituirá por los promedios de las instancias de cada tipo. 
Para obtener la información pertinente iremos calculando en cada instante la distancia entre 
el $y_{gol}$ real y nuestra aproximación del mismo. De esta forma podremos ver si nuestra aproximación converge al valor final de 
la posición en la que el balón atraviesa la línea de meta.
Además introduciremos un ruido de hasta $5\%$, es decir, el valor de la medición generada puede variar hasta en un $5\%$ con 
respecto al valor original de la función en el punto. La forma en la que introducimos el ruido es la siguiente:
\begin{enumerate}
	\item cada punto tiene una probabilidad del $50\%$ de llevar ruido o no
	\item el porcentaje de ruido introducido es un valor del intervalo $[0, 5]$ elegido aleatoriamente con una distribución uniforme
	\item el ruido será sumado o restado también con una probabilidad del $50\%$
\end{enumerate}
Dejaremos de lado a los jugadores y los errores extraordinarios en este experimento ya que nos interesa simplemente evaluar cómo se 
comporta el método con trayectorias simples.  

\subsubsection{Grado del polinomio}
El objetivo de este experimento es analizar el comportamiento de la tasa de efectividad en función del grado del polinomio
con el que aproximamos, mediante el método de cuadrados mínimos, las funciones $x(t)$ e $y(t)$. Nos interesa encontrar 
el grado que maximice la tasa de efectividad. 
\par
Nuestra hipótesis es que el gráfico será cóncavo. Creemos en un primer
momento un aumento del grado implicará un aumento de la efectividad, sin embargo, luego se alcanzará un máximo y se 
empezará a decrecer. Este comportamiento suponemos que se deberá a que considerando instancias que representen curvas variadas 
con un grado demasiado bajo el error de la aproximación será alto, sobre todo en las curvas más irregulares, y con un 
grado demasiado alto la aproximación se volverá más sensible a los errores de medición y perderá capacidad de percibir
la tendencia del disparo.
\par
No conocemos la frecuencia con la que se presentan los diferentes tipos de curvas en los disparo, por lo tanto, la decisión
más prudente consideramos que consiste en asumir una distribución uniforme. Por esta razón no discriminaremos la tasa de efectividad
por tipo de curva con la cual fueron generadas las instancias. Vamos a generar un set de instancias lo más variado y equilibrado
posible que se conformará de la siguiente forma:
\begin{description}
	\item[20] provistas por la cátedra. Este conjunto incluye disparos generados por rectas, parábolas y curvas de tercer grado
	y disparos con ruido.
	\item[40] instancias generadas por nuestro generador de instancias. El ruido agregado a todas las instancias de este
	conjunto es mínimo, entre $0\%$ y $5\%$, de la misma manera que se explicó para el experimento de la evaluación del método. 
	Tenemos 10 instancias generadas con rectas, 10 con parábolas, 10 con curvas de tercer grado y 10 con curvas de cuarto grado.
	\item[40] instancias generadas por nuestro generador de instancias. El ruido agregado a todas las instancias de este 
	conjunto varia uniformemente entre $0\%$ y $20\%$ y fue agregado como se explicó el experimento anterior. El grado de las curvas
	con el que fue generada cada una de las instancias varia también uniformemente entre 0 y 15.
\end{description}
es así que tenemos un conjunto final compuesto por 100 instancias distintas. Todas ellas tienen 25 mediciones. Correremos
las instancias 3 veces considerando sólo las últimas 5, 10 y 15 mediciones para intentar identificar si este parámetro afecta
sustancialmente a la tasa de efectividad conseguida con cada grado utilizado.
\par
Para cada grado la tasa de efectividad final se calculó como el promedio de las tasas de efectividad sobre todas las instancias.

\subsubsection{Puntos a considerar}
Para esta experimentación se desea ver cómo afecta la cantidad de puntos que se consideran para hallar el polinomio de aproximación de 
cuadrados mínimos. Siempre se van a considerar los últimos puntos medidos. En principio, una cantidad baja de puntos podría no ser 
buena ya que no son suficientes para estimar curvas complicadas, tampoco sería bueno que la cantidad sea demasiado alta porque la 
estimación podría ser más propensa a errores, especialmente considerando la presencia de jugadores que pueden alterar el curso de la 
pelota de manera inesperada y que no corresponda con alguna curva "natural''. Como en la experimentación anterior, se usan todo tipo 
de instancias.


%\subsubsection{Errores extraordinarios}

\section{Resultados}

% CONVERGENCIA DEL MÉTODO
\subsection{Experimento de la convergencia del método}
\begin{figure}[H]{}
\centering
\includegraphics[scale=0.5]{graphs/convergenciaMetodo.pdf}
\label{convergenciaMetodo}
\end{figure}

Azul: rectas. Amarillo: cuadráticas. Naranja: cúbicas.

% TASA DE EFICIENCIA EN FUNCIÓN DE LA CANTIDAD DE SUJETOS CONSIDERADOS
\subsection{Experimento de tasa de efectividad en función del grado}
\begin{figure}[H]{}
\centering
\includegraphics[scale=0.5]{graphs/TEvsGrado.pdf}
\label{TEvsGrado}
\end{figure}

% TASA DE EFICIENCIA EN FUNCIÓN DE LA CANTIDAD DE COMPONENTES PRINCIPALES TOMADAS
\subsection{Experimento de tasa de efectividad en función de los puntos considerados}
\begin{figure}[H]{}
\centering
\includegraphics[scale=0.5]{graphs/TEvsMediciones.pdf}
\label{TEvsPac}
\end{figure}

\subsection{Discusión}
%Se incluirá aquí un análisis de los resultados obtenidos en la sección anterior (se analizará
%su validez, coherencia, etc.). Deben analizarse como míınimo los ítems pedidos en el
%enunciado. No es aceptable decir que “los resultados fueron los esperados”, sin hacer
%clara referencia a la teoría la cual se ajustan. Además, se deben mencionar los resul-
%tados interesantes y los casos “patológicos” encontrados.
\subsubsection{Convergencia del método}
Los resultados del experimento arrojaron una convergencia que a simple vista parece cuadrática. Esto nos indica que el método tiene 
consistencia ya que a medida que tiene más información, la pelota está más cerca y tiene más puntos para considerar, la aproximación
es cada vez mejor. No sabemos por qué no es constantemente decreciente la función resultante, quizás se deba al ruido. Deberían
hacerse más experimentos aún para confirmarlo.

\subsubsection{Grado del polinomio}
Como se estimaba, los mejores grados para aproximar las trayectorias de las pelotas corresponden a los polinomios de grado 4, 5 y 6. 
Los polinomios grado 1 y 2 tiene una muy mala aproximación. Mientras que los polinomios de grado mayor o igual a 6 empiezan a mostrar 
una decaída en la tasa de efectividad. Por lo tanto el polinomio final de nuestro aproximación va a el de grado 5.

\subsubsection{Puntos a considerar}
En los resultados se puede apreciar que el valor óptimo se alcanza cuando la cantidad de mediciones es 3, sin embargo, se obtuvieron 
buenos resultados para todos los valores de 4 a 10. Al parecer la cantidad de mediciones no afecta tanto a la tasa de efectividad como 
si lo hace la variación del grado del polinomio. Nuestra aproximación final va a tomar 8 puntos si no hay jugadores y 3 sí los hay.




\section{Conclusiones}

Tendrá sentido aproximar de manera diferente $x(t)$ e $y(t)$. Como nos proveen información diferente quizás sea más prudente sobreestimar por un
lado a uno y sobreestimar por otro al otro.

Estaría bueno hacer varias aproximaciones y comparar unas con otras. Quizás diferentes combinaciones de grados y puntos a considerar no 
nos den mejor o peor información sobre el disparo si no que nos dan diferente información y si la entendemos y aprendemos a combinarla
podemos hacer una aproximación de mayor calidad aún.
\newpage

% Bibliografía
\addcontentsline{toc}{section}{Referencias}
\bibliography{tp3.bib}{}
\bibliographystyle{acm}
\end{document}
