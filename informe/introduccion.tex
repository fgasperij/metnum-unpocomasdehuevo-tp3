
\section{Introducción}

Para la predicción de la posición por donde entrará la pelota, nos basamos en el método de \emph{cuadrados mínimos lineales}. Este método es particularmente adecuado para los casos en que se introducen errores de medición\cite[\emph{3.2}]{heath} como ocurre en las mediciones de la trayectoria de nuestro tp (según se advierte en el enunciado).
\par
Recordemos que la tratectoria de la pelota se puede formalizar mediante la función $p:\mathbb{R} \to \mathbb{R}^2$, $p(t) = (x(t),y(t))$ donde t tomará valores de tiempo discretizado. Nuestra estrategia consiste en calcular los coeficientes de dos polinomios $P_x(t)$ y $P_y(t)$ que aproximen a las funciones $x(t)$ e $y(t)$, respectivamente. 

\par
El planteo formal del método de CM para $P_x(t)$ (para $P_y(t)$ es análogo) es el siguiente:
\par
$
\begin{pmatrix}
  0^G & \cdots    & 0^1 & 1	\\
  \vdots &  \vdots &  \vdots & \vdots  \\ 
  \vdots &  \vdots &  \vdots & \vdots  \\ 
  T^G & \cdots & T^1 & 1\\
\end{pmatrix}
*
\begin{pmatrix}
 coefG\\
 \vdots\\
 coef0\\
\end{pmatrix}
 =
\begin{pmatrix}
  x(0)\\
  \vdots \\
  \vdots \\
  x(T)\\ 

\end{pmatrix}
$
\par
donde la cantidad de filas se corresponde con la cantidad de instantes \emph{T} que consideramos de la trayectoria de un tiro en particular (comenzando desde el instante cero) y donde \emph{G} es el grado del polinomio con el que estimamos a $x(t)$ (lo elegimos siempre menor a la cantidad de instantes considerados).
\par
Así el polinomio que aproxima a $x(t)$ sería $P_x(t)=coefG*t^G + \cdots + coef1*t + coef0$
\par
Como en nuestro modelo, la pelota se atajará sobre la recta x=125, calcularemos entonces la raiz $t_{x=125}$ del polinomio $\hat{P}_x(t) = P_x(t) -125$; de manera que $P_y(t_{x=125})$ será la posición que nuestro arquero intentará cubrir en caso de encontrarse entre los límites del arco.
