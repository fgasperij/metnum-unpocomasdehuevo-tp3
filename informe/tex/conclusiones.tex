\section{Conclusiones}

Luego de haber experimentado y analizado el comportamiento de nuestro método conseguimos una confianza bastante sólida sobre el método
de cuadrados mínimos. Sin embargo, creemos que es un problema que tiene muchas aristas y que da aún mucho por investigar. Creemos
que algo que beneficiaría mucho a lal optimización del método es un análisis físico de lo que sucede en los disparos, las fuerzas
que intervienen y cómo afectan al balón. Además, nos parece interesante explorar si es mejor aproximar de manera diferente $x(t)$ e $y(t)$. 
Como nos proveen información diferente, de una extraemos $t_{gol}$ y de la otra $y_{gol}$ quizás sea más prudente sobreestimar por un
lado a uno y sobreestimar por otro al otro. También nos parece interesante estudiar si mejora la tasa de efectividad
hacer varias aproximaciones y comparar unas con otras. Quizás diferentes combinaciones de grados y puntos a considerar no 
nos den mejor o peor información sobre el disparo si no que nos dan diferente información y si la entendemos y aprendemos a combinarla
podemos hacer una aproximación de mayor calidad aún.
El problema sigue abierto con muchos caminos por explorar, sin embargo, esperamos haber echado algo de luz para los investigadores venideros.