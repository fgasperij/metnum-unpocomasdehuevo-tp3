\section{Desarrollo}

\subsection{Descripción del método}
Dado un instante $t$ nosotros contamos con $(t+1)$ mediciones de la posición de la pelota que se corresponden con los intantes 
0, 1, ..., $t-1$, $t$. 
\begin{center}
  \begin{tabular}{l | l*{3}{c}r}
  t              & 0 & 1 & ... & (t-1) & t\\
  \hline
  x(t)	       & x(0) & x(1) & ... & x(t-1) & x(t)\\
  \hline
  y(t)	       & y(0) & y(1) & ... & y(t-1) & y(t)\\
  \end{tabular} 
\end{center}
De esta forma la posición de la pelota queda descripta por una función $p:\mathbb{R} \to \mathbb{R}^2$ tal que $p(t) = (x(t),y(t))$.
Nuestro procedimiento para decidir el sentido y la magnitud del desplazamiento del arquero para un instante $t$ consiste de los siguientes
pasos:
\begin{enumerate}
 \item aproximamos la función $x(t)$ con cuadrados mínimos y obtenemos $P_x(t)$
 \item calculamos una raíz de $x'(t) = x(t) - x_{posicionDelArco}$ para conocer en qué instante $t_{raiz}$ atraviesa la línea de meta
 \item aproximamos la función $y(t)$ con cuadrados mínimos y obtenemos $P_y(t)$
 \item evaluamos $P_y(t)$ en $t_{raiz}$ para obtener $P_y(t_{raiz}) = y_{aproximacion}$
 \item calculamos el desplazamiento óptimo del arquero asumiendo que el disparo ingresará por $y_{aproximacion}$
\end{enumerate}

La aproximación de la función $x(t)$ (para $y(t)$ es análogo) mediante el método de cuadrados mínimos la hacemos resolviendo el sistema:
\begin{displaymath}
\begin{pmatrix}
  0^n & \cdots    & 0^1 & 1	\\
  \vdots &  \vdots &  \vdots & \vdots  \\ 
  \vdots &  \vdots &  \vdots & \vdots  \\ 
  t^n & \cdots & t^1 & 1\\
\end{pmatrix}
*
\begin{pmatrix}
 a_n\\
 \vdots\\
 a_0\\
\end{pmatrix}
 =
\begin{pmatrix}
  x(0)\\
  \vdots \\
  \vdots \\
  x(t)\\ 
\end{pmatrix} 
\end{displaymath}

donde la cantidad de filas se corresponde con la cantidad de instantes $t$ que consideramos de la trayectoria y $n$ es el grado del 
polinomio con el que estimamos a $x(t)$. Para resolver el sistema utilizamos eliminación gaussiana por ser un métodos de simple
implementación. Una vez resuelto el sistema utilizamos los $a_i$ para construir el polinomio que aproxima a $x(t)$ de la siguiente forma:
\begin{displaymath}
P_x(t)=a_n*t^n + \cdots + a_1*t + a_0 
\end{displaymath}
Sabemos que $P_x(t)$ es el polinomio de grado $n$ que minimiza la suma del error cuadrático con respecto a las mediciones. 
\par
La línea de meta se encuentra en un valor fijo de $x$ al que nos referiremos como $x_{posicionDelArco}$. El arquero siempre está 
situado sobre ese valor $x_{posicionDelArco}$ y se mueve a lo largo del eje $y$. Nuestra estimación debe ser el valor de $y$ por el
cual el balón atravesará la línea de meta. Si $p(t) = (x(t), y(t))$ describe la trayectoria del balón y asumimos que existe un $t_{gol}$ para el 
cual:
\begin{displaymath}
  p(t_{gol}) = (x(t_{gol}), y(t_{gol})) = (x_{posicionDelArco}, y(t_{gol}))
\end{displaymath}
a nosotros nos interesa aproximar $y(t)$ para poder evaluarla en $t_{gol}$ y así conocer en qué valor de $y$ el balón atravesará la línea
de meta. Para aproximar $y(t)$ utilizamos el método de cuadrados mínimos ya explicado para aproximar $x(t)$ y obtenemos un $P_y(t)$. Luego,
vamos a calcular $t_{gol}$ con nuestro $P_x(t) \approx x(t)$:
\begin{align}
    P_x(t) = x_{posicionDelArco}  \Longleftrightarrow t = t_{gol}\\
    P_x(t) - x_{posicionDelArco} = 0 \Longleftrightarrow t = t_{gol}\\
\end{align}
Tomando $P'_x(t) = P_x(t) - x_{posicionDelArco}$ sabemos que si tiene raíces reales mayores al último instante $t$ de la trayectoria 
entonces $t_{gol}$ es una de ellas. Para encontrarla utilizamos el método de bisección \cite[]{BoostSite}. Una vez que obtenemos 
$t_{gol}$ evaluamos: 
\begin{displaymath}
  P_y(t_{gol}) = y_{aproximacion}
\end{displaymath}
$y_{aproximacion}$ será la posición que el arquero intentará cubrir.
\par

\subsection{Elección de las aproximaciones}
Un caso particular es cuando contamos con sólo una sólo medición de la posición del balón ya que no podemos aplicar cuadrados mínimos. En
ese caso aproximamos por la recta constante. Es decir, que cuando nuestra única medición es $p(0) = (x, y)$ nuestra estimación es que la 
pelota atravesará la línea de meta en $y$. Esta decisión no tiene ningún soporte matemático ni experimental, simplemente nos resultó
intuitiva.
\par
Como las mediciones de la posición del balón pueden tener errores propios de los instrumentos utilizados o ruidos de fuentes circunstanciales
evitamos aproximar con el polinomio interpolador ya que hacerlo generaría que el mismo, $P_{x/y}(t)$, sea sensible a estos errores. Es por esta
razón que cuadrados mínimos, con un grado menor al interpolador, resulta en una aproximación más fiable de la función original aunque 
no atraviese todos los puntos obtenidos por las mediciones. Dados $n$ puntos $(x_i, y_i)$ sabemos que siempre existe un polinomio único 
que los interpola de grado menor o igual a $(n-1)$. Por lo tanto, como en el instante $t$ contamos con una trayectoria de $t+1$
posiciones nuestro $P_{x/y}(t)$ siempre lo construimos con un grado menor a $t$.
\par
Cuando intentamos calcular la raíz de $P'_x(t)$ que se corresponde con $t_{gol}$ y no lo conseguimos simplemente descartamos ese $P_x(t)$
e intentamos con otro de grado diferente. Sin embargo, un polinomio tiene un número de raíces igual a su grado por lo tanto cuanto más 
alto el grado mayor la cantidad de raíces y esto redunda en una mayor dificultad para encontrar la raíz correspondiente a $t_{gol}$. Para
intentar compensar este efecto sólo tomamos las últimas mediciones que efectivamente se acercan a la línea de meta. Si nuestras mediciones
se corresponden con los puntos $(x_0, y_0)$, $(x_1, y_1)$, ..., $(x_t, y_t)$ entonces sólo consideraremos los puntos $p_i$ tales que:
\begin{displaymath}
 (\forall j \in \mathbb{N}) j > i \Rightarrow x_j < x_{j-1}
\end{displaymath}

\subsection{Errores de medición}\label{ssec:errores_de_medicion}
En general asumimos que los errores de medición son pequeños, sin embargo, puede ocurrir que debido a una circunstancia extraordinaria
se registren algunos valores con un error muy grande. Estos outliers deben reconocerse y tratarse de un modo particular para que perturben en
la menor medida posible nuestras aproximaciones. En la sección de experimentación se probará un método para reconocerlos y nuestro 
tratamiento de los mismos será ignorarlos, es decir, continuar aproximando como si esa medición no hubiera sido realizada. Si esa medición
fue la correspondiente a el instante $t$, cuando obtengamos la medición del instante $t+1$ lo que haremos será extrapolar cuadráticamente
la medición del instante $t$ para no sumarle velocidad innecesaria a nuestras aproximaciones.

\subsection{Jugadores rivales}
La posición de los jugadores rivales juega un papel muy importante al momento de aproximar $p(t)$ ya que un jugador puede 
patear la pelota y cambiar por completo su dirección y su velocidad. En primer lugar, si el balón pasa a una distancia menor a un umbral
determinado de antemano de la posición de un jugador a la próxima medición no le aplicamos la detección de errores de medición
extraordinarios que explicamos en \ref{ssec:errores_de_medicion}. Además una vez que el balón pasa por un jugador vamos a 
dejar de considerar las mediciones anteriores ya que si el jugador patéo el balón es efectivamente un disparo nuevo y por lo tanto las
mediciones anteriores no aportan información alguna.

















