\section{Desarrollo}

\subsection{Descripción del método}
Dado un instante $t$ nosotros contamos con $(t+1)$ mediciones de la posición de la pelota que se corresponden con los intantes 
0, 1, ..., $t-1$, $t$. 
\begin{center}
  \begin{tabular}{l | l*{3}{c}r}
  t              & 0 & 1 & ... & (t-1) & t\\
  \hline
  x(t)	       & x(0) & x(1) & ... & x(t-1) & x(t)\\
  \hline
  y(t)	       & y(0) & y(1) & ... & y(t-1) & y(t)\\
  \end{tabular} 
\end{center}
De esta forma la posición de la pelota queda descripta por una función $p:\mathbb{R} \to \mathbb{R}^2$ tal que $p(t) = (x(t),y(t))$.
Nuestro procedimiento para decidir el sentido y la magnitud del desplazamiento del arquero para un instante $t$ consiste de los siguientes
pasos:
\begin{enumerate}
 \item aproximamos la función $x(t)$ con cuadrados mínimos obteniendo $P_x(t)$
 \item calculamos una raíz de $Q_x(t) = P_x(t) - x_{posicionDelArco}$ para conocer en qué instante $t_{raiz}$ atraviesa la línea de meta
 \item aproximamos la función $y(t)$ con cuadrados mínimos obteniendo $P_y(t)$
 \item evaluamos $P_y(t)$ en $t_{raiz}$ para obtener $P_y(t_{raiz}) = y_{aproximacion}$
 \item calculamos el desplazamiento óptimo del arquero asumiendo que el disparo ingresará por $y_{aproximacion}$
\end{enumerate}

El planteo formal del método de cuadrados mínimos para la aproximación de la función $x(t)$ (para $y(t)$ es análogo) es el siguiente:
\begin{displaymath}
\begin{pmatrix}
  0^n & \cdots    & 0^1 & 1	\\
  \vdots &  \vdots &  \vdots & \vdots  \\ 
  \vdots &  \vdots &  \vdots & \vdots  \\ 
  t^n & \cdots & t^1 & 1\\
\end{pmatrix}
*
\begin{pmatrix}
 a_n\\
 \vdots\\
 a_0\\
\end{pmatrix}
 =
\begin{pmatrix}
  x(0)\\
  \vdots \\
  \vdots \\
  x(t)\\ 
\end{pmatrix} 
\end{displaymath}

donde la cantidad de filas se corresponde con la cantidad de instantes $t$ que consideramos de la trayectoria y $n$ es el grado del 
polinomio con el que estimamos a $x(t)$. Resolviendo el sistema de ecuaciones normales correspondiente al sistema anterior, obtenemos los coeficientes de $P_x(t)$. Para resolverlo utilizamos eliminación gaussiana por ser un métodos de simple
implementación. Siempre podemos aplicar eliminación gaussiana porque el grado del polinomio que utilizamos para aproximar por cuadrados mínimos, $g$,  es como máximo menor o igual a la cantidad de mediciones que tenemos menos 2. Esto lo decidimos así para nunca utilizar un polinomio interpolador, ya que el mismo sería sensible hasta a los errores de medición.
\begin{displaymath}
	g \leq \#mediciones - 2 \Longrightarrow \#columnas(A) \leq \#filas(A) - 2
\end{displaymath}
por lo tanto la matriz $A$ va a tener rango columna completo.
 La solución son los $a_i$ con los que construimos el polinomio que aproxima a $x(t)$ de la siguiente forma:
\begin{displaymath}
	P_x(t)=a_n*t^n + \cdots + a_1*t + a_0 
\end{displaymath}
Así resulta que $P_x(t)$ es el polinomio de grado $n$ que minimiza la suma del error cuadrático con respecto a las mediciones. 
\par
La línea de meta se encuentra en un valor fijo de $x$ al que nos referiremos como $x_{posicionDelArco}$. El arquero siempre está 
situado sobre ese valor $x_{posicionDelArco}$ y se mueve a lo largo del eje $y$. Nuestra estimación debe ser el valor de $y$ por el
cual el balón atravesará la línea de meta. Si $p(t) = (x(t), y(t))$ describe la trayectoria del balón y asumimos que existe un $t_{gol}$ para el 
cual:
\begin{displaymath}
  p(t_{gol}) = (x(t_{gol}), y(t_{gol})) = (x_{posicionDelArco}, y(t_{gol}))
\end{displaymath}
a nosotros nos interesa aproximar $y(t)$ para poder evaluarla en $t_{gol}$ y así conocer en qué valor de $y$ el balón atravesará la línea
de meta. Para aproximar $y(t)$ utilizamos el método de cuadrados mínimos ya explicado para aproximar $x(t)$ y obtenemos un $P_y(t)$. Luego,
vamos a calcular $t_{gol}$ con nuestro $P_x(t) \approx x(t)$:
\begin{align}
    P_x(t) = x_{posicionDelArco}  \Longleftrightarrow t = t_{gol}\\
    P_x(t) - x_{posicionDelArco} = 0 \Longleftrightarrow t = t_{gol}
\end{align}
Tomando $Q_x(t) = P_x(t) - x_{posicionDelArco}$ sabemos que si tiene raíces reales mayores al último instante $t$ de la trayectoria 
entonces $t_{gol}$ es una de ellas. Para encontrarla utilizamos el método de bisección \cite[]{BoostSite}. 
El intervalo inicial $[a,b]$ siempre es el mismo con $a = 0$ y $b = 10000$. Sabemos que $Q_x(0) > 0$ porque el disparo siempre se origina
dentro del campo de juego. Necesitamos un valor $b$ para el cual sepamos que $Q_x(b) < 0$. Esto es lo mismo que pedir un instante $t$
para el cual la posición de la pelota se encuentre del otro lado de la línea de meta según nuestra aproximación por cuadrados mínimos. 
Tomamos un valor lo suficientemente grande como para asegurarnos que el disparo haya terminado antes del mismo. Es cierto que esto
no nos asegura que $Q_x(b) < 0$ porque podría ser que $Q_x(t)$ tuviera más de una raíz real mayor a 0. Si nos encontraramos con este caso
lo descartaríamos. Sin embargo, consideramos que las probabilidades son bajas ya que para que el disparo termine en gol debe ir acercándose
a la línea de meta hasta finalmente cruzarla. Que se vaya acercando hasta finalmente cruzarla lo traducimos a que $Q_x(t)$ debe decrecer hasta finalmente
cruzar el eje x. Que $Q_x(t)$ vuelva a crecer y así tener otra raíz real intuimos que no es probable dado que no hay mediciones que se lo propongan,
las últimas deben hacerla decrecer hasta cruzar la línea de meta. 
Una vez que obtenemos 
$t_{gol}$ evaluamos: 
\begin{displaymath}
  P_y(t_{gol}) = y_{aproximacion}
\end{displaymath}
$y_{aproximacion}$ será la posición que el arquero intentará cubrir.
\par

\subsection{Elección de las aproximaciones}
Un caso particular es cuando contamos con sólo una sólo medición de la posición del balón ya que no podemos aplicar cuadrados mínimos. En
ese caso aproximamos por la recta constante. Es decir, que cuando nuestra única medición es $p(0) = (x, y)$ nuestra estimación es que la 
pelota atravesará la línea de meta en $y$. Esta decisión no tiene ningún soporte matemático ni experimental, simplemente nos resultó
intuitiva.
\par
Como las mediciones de la posición del balón pueden tener errores propios de los instrumentos utilizados o ruidos de fuentes circunstanciales
evitamos aproximar con el polinomio interpolador ya que hacerlo generaría que el mismo, $P_{x/y}(t)$, sea sensible a estos errores. Es por esta
razón que cuadrados mínimos, con un grado menor al interpolador, resulta en una aproximación más fiable de la función original aunque 
no atraviese todos los puntos obtenidos por las mediciones. Dados $n$ puntos $(x_i, y_i)$ sabemos que siempre existe un polinomio único 
que los interpola de grado menor o igual a $(n-1)$. Por lo tanto, como en el instante $t$ contamos con una trayectoria de $t+1$
posiciones nuestro $P_{x/y}(t)$ siempre lo construimos con un grado menor a $t$.
\par
Cuando intentamos calcular la raíz de $Q_x(t)$ que se corresponde con $t_{gol}$ y no lo conseguimos simplemente descartamos ese $P_x(t)$
e intentamos con otro de grado diferente. Sin embargo, un polinomio tiene un número de raíces igual a su grado por lo tanto cuanto más 
alto el grado mayor la cantidad de raíces y esto redunda en una mayor dificultad para encontrar la raíz correspondiente a $t_{gol}$. Para
intentar compensar este efecto sólo tomamos las últimas mediciones que efectivamente se acercan a la línea de meta. Si nuestras mediciones
se corresponden con los puntos $(x_0, y_0)$, $(x_1, y_1)$, ..., $(x_t, y_t)$ entonces sólo consideraremos los últimos $p_i$ puntos tales que:
\begin{displaymath}
x_i > x_{i-1}
\end{displaymath}
es decir, los últimos $p_i$ que efectivamente se están acercando a la meta.

\subsection{Errores de medición}\label{ssec:errores_de_medicion}
En general asumimos que los errores de medición son pequeños, sin embargo, puede ocurrir que debido a una circunstancia extraordinaria
se registren algunos valores con un error muy grande. Estos outliers deben reconocerse y tratarse de un modo particular para que perturben en
la menor medida posible nuestras aproximaciones. Nuestro 
tratamiento de los mismos será ignorarlos, es decir, continuar aproximando como si esa medición no hubiera sido realizada.  En cada instante $t$ 
aproximamos el $y_gol$ para mover al arquero y además calculamos: 
\begin{displaymath}
	p_{t+1} = (x_{t+1}, y_{t+1})
\end{displaymath}
En el instante $t+1$ comparamos la medición que recibimos con nuestra previa aproximación $p_{t+1}$. Calculamos el error relativo total de la siguiente
forma:
\begin{displaymath}
	e_{relativoX} = \frac{|xAproximacion_{t+1} - xMedicion_{t+1}|}{xMedicion_{t}}
\end{displaymath}
\begin{displaymath}
	e_{relativoTotal} = \frac{e_{relativoX} + e_{relativoY}}{2}
\end{displaymath}
Si el error relativo total resulta ser mayor al $80\%$ entonces lo que hacemos es descartar esa medición y utilizar nuestra previa aproximación. Luego,
en el instante $t+2$ utilizaremos como medición del instante $t+1$ una interpolación cuadrática entre los valores de la medición del instante $t$ y 
la del instante $t+2$. Es necesario aclarar que la detección de outliers sólo se considera válida si evaluamos que el cambio no puede deberse a que un jugador rival
alteró la trayectoria del balón. Consideramos que un jugador puede haber alterado la trayectoria del balón si el mismo pasa a una distancia menor a 7, la misma
que se utiliza para considerar si el arquero atajó o no el disparo.

\subsection{Jugadores rivales}
La posición de los jugadores rivales juega un papel muy importante al momento de aproximar $p(t)$ ya que un jugador puede 
patear la pelota y cambiar por completo su dirección y su velocidad. En primer lugar, si el balón pasa a una distancia menor a un umbral
determinado de antemano de la posición de un jugador a la próxima medición no le aplicamos la detección de errores de medición
extraordinarios que explicamos en \ref{ssec:errores_de_medicion}. Además una vez que el balón pasa por un jugador vamos a 
dejar de considerar las mediciones anteriores ya que si el jugador patéo el balón es efectivamente un disparo nuevo y por lo tanto las
mediciones anteriores no aportan información alguna.

















