\subsection{Discusión}
%Se incluirá aquí un análisis de los resultados obtenidos en la sección anterior (se analizará
%su validez, coherencia, etc.). Deben analizarse como míınimo los ítems pedidos en el
%enunciado. No es aceptable decir que “los resultados fueron los esperados”, sin hacer
%clara referencia a la teoría la cual se ajustan. Además, se deben mencionar los resul-
%tados interesantes y los casos “patológicos” encontrados.
\subsubsection{Convergencia del método}
Se puede ver claramente que el método converge para los 3 tipos de instancias utilizados. A partir de el instante 6
las 3 curvas son monótonas decrecientes, esto quiere decir que a partir de ese momento las aproximaciones realizadas 
se acercan ininterrumpidamente a $y_gol$. En los primero 5 instantes las curvas correspondientes a instancias 
de rectas y a curvas de tercer grado no presentan conductas raras, sin embargo, la de parábolas sí. 
Entre los instantes 2 y 6 crece linealmente. No estamos seguros a qué se debe, nuestra única conjetura es que el error
pudo haber sido casualmente creciente en ese intervalo. Debería conducirse un nuevo experimento con instancias de ese tipo
para verificar si ese comportamiento repite.
\par
Creemos que los resultados son buenos pero no óptimos. Verificamos que el método es razonable y funciona bien para 
disparos simples, no obstante, creemos que es necesario en futuros análisis una experimentación más extensiva que incluya
disparos con jugadores que intervengan, con más ruido y generados por curvas de un grado mayor, más impredecibles. Ya que
no podemos asegurar una convergencia razonable para ésos otros tipos de instancias. Además, si bien no calculamos
en el gráfico la velocidad de convergencia intuimos que es lineal y se puede mejorar. Un primer paso en esta dirección
es utilizar los parámetros optimizados en los experimentos subsiguientes.

\subsubsection{Grado del polinomio}
Las 4 curvas resultantes son cóncavas lo cual coincide con nuestras hipótesis originales. Independientemente de la cantidad
de mediciones consideradas siempre un grado obtuvo la mayor tasa de efectividad como máximo absoluto, ya que la curva decrece
en los dos sentidos a partir de él. No se puede apreciar en el gráfico la concavidad de la curva verde correspondiente a 15 mediciones
pero intuimos que se debe a que el gráfico termina cerca de su máximo. Para corroborar esto se requiere realizar otro experimento
con más mediciones.
\par
Un patrón que reconocemos en las 4 curvas es que su máximo siempre se ubica en un grado aproximadamente dos unidades menor a la
cantidad de mediciones tomadas. Ese es un momento muy particular ya que se corresponde con el grado máximo que puede aproximar
una cierta cantidad de mediciones sin interpolarla. A primera vista esto corroboraría nuestra hipótesis sobre el perjuicio que
conlleva aproximar con un polinomio que interpole las mediciones pero no lo consideramos conclusivo. Lo que no puede apreciarse 
en el gráfico es el detrimento en la tasa de efectividad que esperábamos para polinomios de grados altos. Creemos que esto se 
debe a que no utilizamos grados suficientemente altos, quizás con grados mayores a 10 ya podemos apreciar esto, sin embargo, 
es necesario realizar nuevos experimentos para continuar el estudio de esa hipótesis.
\par
En promedio el grado que mayor tasa de efectividad obtuvo fue el 5. Consideramos que si bien quizás sea el grado más prudente
para obtener una tasa de efectividad razonable dista mucho de ser el óptimo para cada circunstancia. Claramente existen
opciones superiores en cada circunstancia en particular pero para optimizar este aspecto creemos que se debería conocer
más la distribución de la frecuencia de los tipos de disparos o aunque sea intentar descifrar a una curva de qué grado
se parecen más las mediciones obtenidas y optimizar el grado utilizado para la misma.

\subsubsection{Puntos a considerar}
Los resultados obtenidos se correlacionan bastante con los del experimento anterior ya que se desprende la misma
relación entre el pico de tasa de efectividad de las curvas y el punto en el que el grado es aproximadamente
dos unidades menor a la cantidad de mediciones consideradas. Vemos que hasta cierto momento las curvas son prácticamente iguales
y se van desprendiendo de a una del resto. Esto tiene sentido ya que hasta que la cantidad de mediciones consideradas no supera
al grado utilizado siempre se está utilizando un polinomio interpolador. Por lo tanto, una curva va a desprenderse del resto
en el momento en el que las mediciones superen su grado y eso es precisamente lo que se observa en el gráfico.
\par
La cantidad de mediciones que tiene una buena tasa de efectividad para todos los grados es 6. Creemos que esto tiene sentido ya que 
es un valor que está cerca del promedio de los grados considerados. Además, atribuimos el decrecimiento que se produce en
las curvas correspondientes a instancias de tercer y quinto grado a que a medida que se consideran más mediciones las más
viejas contienen información sobre el disparo que no sólo los polinomios de grado tan bajo no saben utilizar sino que 
encima las perjudica ya que las corre innecesariamente.

