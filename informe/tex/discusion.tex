\subsection{Discusión}
%Se incluirá aquí un análisis de los resultados obtenidos en la sección anterior (se analizará
%su validez, coherencia, etc.). Deben analizarse como míınimo los ítems pedidos en el
%enunciado. No es aceptable decir que “los resultados fueron los esperados”, sin hacer
%clara referencia a la teoría la cual se ajustan. Además, se deben mencionar los resul-
%tados interesantes y los casos “patológicos” encontrados.
\subsubsection{Grado del polinomio}
Como se estimaba, los mejores grados para aproximar las trayectorias de las pelotas corresponden a los polinomios de grado 4, 5 y 6. Los polinomios grado 1 y 2 tiene una muy mala aproximación. Mientras que los polinomios de grado mayor o igual a 6 empiezan a mostrar una decaída en la tasa de efectividad. Por lo tanto el polinomio final de nuestro aproximación va a el de grado 5.

\subsubsection{Puntos a considerar}
En los resultados se puede apreciar que el valor óptimo se alcanza cuando la cantidad de mediciones es 3, sin embargo, se obtuvieron buenos resultados para todos los valores de 4 a 10. Al parecer la cantidad de mediciones no afecta tanto a la tasa de efectividad como si lo hace la variación del grado del polinomio. Nuestra aproximación final va a tomar 8 puntos si no hay jugadores y 3 sí los hay.


