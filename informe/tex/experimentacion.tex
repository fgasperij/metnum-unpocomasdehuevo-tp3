\section{Experimentación}

\subsection{Discusión sobre el método y sus variables}
La efectividad del método estará dada por la calidad de la aproximación del lugar por el cual el disparo llegará a la línea de meta. 
Nuestra aproximación de $x(t)$ nos da el instante en el cual el balón atravesará la línea de meta y luego nuestra aproximación
de $y(t)$ nos dirá el lugar por dónde lo hará. Las variables principales que entran en juego al aproximar una función con 
cuadrados mínimos son las mediciones a considerar, es decir los puntos, y el grado del polinomio con el cual se aproximará la función.
Creemos que un análisis físico sobre las diferentes fuerzas que puedan afectar a la dirección y aceleración del balón puede indicar 
más firmemente qué grados tiene más sentido utilizar para cuadrados mínimos. Sin embargo, el mismo cae fuera del alcance del presente 
trabajo por lo cual nos limitaremos a experimentar con el mayor rango de grados posible y quizás los resultados nos arrojen alguna pista.
Con respecto a qué puntos de la trayectoria utilizar intuimos que no todos los puntos de la trayectoria proveen información de la misma 
calidad. Es por eso que destinaremos un experimento entero a evaluar qué tasa de efectividad obtenemos tomando diferentes subconjuntos 
de puntos de la trayectoria.

\subsection{Bases generales}
Criterios que aplicamos a todos los experimentos:
\begin{itemize}
	\item Para determinar si cierta combinación de parámetros tuvo una mejor performance que otra, vamos a basarnos exclusivamente en la tasa 
	de efectividad de atajadas del arquero. Solamente vamos a considerar si nuestra aproximación converge a la posición final en el experimento para 
	evaluar el método.
	\item Para generar instancias basadas en una función polinómica de grado $g$ tomamos un punto al azar del campo de juego y otro al azar entre los dos postes 
del arco. Luego tomamos tantos puntos al azar sobre el campo de juego entre el primer punto y la línea de meta hasta alcanzar una cantidad
de puntos igual a $\#p = g + 1$. Con esos puntos generamos el polinomio de Lagrange asociado y con el mismo generamos el resto de los puntos
que queramos para nuestra trayectoria.
	\item El grado del polinomio utilizado en los experimentos no est
	\begin{enumerate}
		\item
	\end{enumerate}
\end{itemize}

\subsection{Experimentos}

\subsubsection{Evaluación del método}
El objetivo de esta experimentación es evaluar si la aproximación de nuestro método realmente converge al valor final de 
la posición $y$ por la cual el balón atraviesa la línea de meta. Lo haremos con lo que consideramos disparos básicos de una situación
real de juego. Éstos comprenden disparos con trayectorias de tres tipos: rectas, parábolas y curvas generadas por polinomios de tercer
grado. Separaremos en esos tres tipos y generaremos para cada uno de ellos 100 instancias aleatorias. Todas ellas compartirán 
una cantidad de mediciones igual a 30 y el gráfico que presentaremos se constituirá por los promedios de las instancias de cada tipo. 
Para obtener la información pertinente iremos calculando en cada instante la distancia entre 
el $y_{gol}$ real y nuestra aproximación del mismo. De esta forma podremos ver si nuestra aproximación converge al valor final de 
la posición en la que el balón atraviesa la línea de meta.
Además introduciremos un ruido de hasta $5\%$, es decir, el valor de la medición generada puede variar hasta en un $5\%$ con 
respecto al valor original de la función en el punto. La forma en la que introducimos el ruido es la siguiente:
\begin{enumerate}
	\item cada punto tiene una probabilidad del $50\%$ de llevar ruido o no
	\item el porcentaje de ruido introducido es un valor del intervalo $[0, 5]$ elegido aleatoriamente con una distribución uniforme
	\item el ruido será sumado o restado también con una probabilidad del $50\%$
\end{enumerate}
Dejaremos de lado a los jugadores y los errores extraordinarios en este experimento ya que nos interesa simplemente evaluar cómo se 
comporta el método con trayectorias simples.  

\subsubsection{Grado del polinomio}
El objetivo de este experimento es analizar el comportamiento de la tasa de efectividad en función del grado del polinomio
con el que aproximamos, mediante el método de cuadrados mínimos, las funciones $x(t)$ e $y(t)$. Nos interesa encontrar 
el grado que maximice la tasa de efectividad. 
\par
Nuestra hipótesis es que el gráfico será cóncavo. Creemos en un primer
momento un aumento del grado implicará un aumento de la efectividad, sin embargo, luego se alcanzará un máximo y se 
empezará a decrecer. Este comportamiento suponemos que se deberá a que considerando instancias que representen curvas variadas 
con un grado demasiado bajo el error de la aproximación será alto, sobre todo en las curvas más irregulares, y con un 
grado demasiado alto la aproximación se volverá más sensible a los errores de medición y perderá capacidad de percibir
la tendencia del disparo.
\par
No conocemos la frecuencia con la que se presentan los diferentes tipos de curvas en los disparo, por lo tanto, la decisión
más prudente consideramos que consiste en asumir una distribución uniforme. Por esta razón no discriminaremos la tasa de efectividad
por tipo de curva con la cual fueron generadas las instancias. Vamos a generar un set de instancias lo más variado y equilibrado
posible que se conformará de la siguiente forma:
\begin{description}
	\item[20] provistas por la cátedra. Este conjunto incluye disparos generados por rectas, parábolas y curvas de tercer grado
	y disparos con ruido.
	\item[40] instancias generadas por nuestro generador de instancias. El ruido agregado a todas las instancias de este
	conjunto es mínimo, entre $0\%$ y $5\%$, de la misma manera que se explicó para el experimento de la evaluación del método. 
	Tenemos 10 instancias generadas con rectas, 10 con parábolas, 10 con curvas de tercer grado y 10 con curvas de cuarto grado.
	\item[40] instancias generadas por nuestro generador de instancias. El ruido agregado a todas las instancias de este 
	conjunto varia uniformemente entre $0\%$ y $20\%$ y fue agregado como se explicó el experimento anterior. El grado de las curvas
	con el que fue generada cada una de las instancias varia también uniformemente entre 0 y 15.
\end{description}
es así que tenemos un conjunto final compuesto por 100 instancias distintas. Todas ellas tienen 25 mediciones. Correremos
las instancias 3 veces considerando sólo las últimas 5, 10 y 15 mediciones para intentar identificar si este parámetro afecta
sustancialmente a la tasa de efectividad conseguida con cada grado utilizado.
\par
Para cada grado la tasa de efectividad final se calculó como el promedio de las tasas de efectividad sobre todas las instancias.

\subsubsection{Puntos a considerar}
Para esta experimentación se desea ver cómo afecta la cantidad de puntos que se consideran para hallar el polinomio de aproximación de 
cuadrados mínimos. Siempre se van a considerar los últimos puntos medidos. En principio, una cantidad baja de puntos podría no ser 
buena ya que no son suficientes para estimar curvas complicadas, tampoco sería bueno que la cantidad sea demasiado alta porque la 
estimación podría ser más propensa a errores, especialmente considerando la presencia de jugadores que pueden alterar el curso de la 
pelota de manera inesperada y que no corresponda con alguna curva "natural''. Como en la experimentación anterior, se usan todo tipo 
de instancias.


%\subsubsection{Errores extraordinarios}
