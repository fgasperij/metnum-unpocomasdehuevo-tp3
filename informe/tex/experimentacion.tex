\section{Experimentación}

\subsection{Discusión sobre el método y sus variables}
La efectividad del método estará dada por la calidad de la aproximación del lugar por el cual el disparo llegará a la línea de meta. 
Nuestra aproximación de $x(t)$ nos da el instante en el cual el balón atravesará la línea de meta y luego nuestra aproximación
de $y(t)$ nos dirá el lugar por dónde lo hará. Las variables principales que entran en juego al aproximar una función con 
cuadrados mínimos son las mediciones a considerar, los puntos, y el grado del polinomio con el cual se aproximará la función.


De qué grado será el polinomio con el cual aproximaremos a las funciones $x(t)$ e $y(t)$. Alguno tiene más sentido que otro. ¿Tiene
sentido que sea el mismo para las dos?

¿Tiene sentido considerar todos los puntos de la trayectoria? ¿Algunos puntos contienen más información que otros?

¿Tiene sentido hacer varias aproximaciones y comparar? ¿Qué información nos estaría brindando cada una?



\subsection{Bases generales}
Para determinar si cierta combinación de parámetros tuve una mejor performance que otra vamos a basarnos exclusivamente en la tasa de
efectividad de atajadas del arquero. Generamos instancias de 3 tipos: rectas, cuadráticas y aleatorias.

\subsection{Experimentos}

\subsubsection{Grado del polinomio}
\subsubsection{Puntos a considerar}
\subsubsection{Errores extraordinarios}
\subsubsection{Jugadores}
\subsubsection{Atajadas con memoria}