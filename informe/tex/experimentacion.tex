\section{Experimentación}

\subsection{Discusión sobre el método y sus variables}
La efectividad del método estará dada por la calidad de la aproximación del lugar por el cual el disparo llegará a la línea de meta. 
Nuestra aproximación de $x(t)$ nos da el instante en el cual el balón atravesará la línea de meta y luego nuestra aproximación
de $y(t)$ nos dirá el lugar por dónde lo hará. Las variables principales que entran en juego al aproximar una función con 
cuadrados mínimos son las mediciones a considerar, es decir los puntos, y el grado del polinomio con el cual se aproximará la función.
Creemos que un análisis físico sobre las diferentes fuerzas que puedan afectar a la dirección y aceleración del balón puede indicar 
más firmemente qué grados tiene más sentido utilizar para cuadrados mínimos. Sin embargo, el mismo cae fuera del alcance del presente 
trabajo por lo cual nos limitaremos a experimentar con el mayor rango de grados posible y quizás los resultados nos arrojen alguna pista.
Con respecto a qué puntos de la trayectoria utilizar intuimos que no todos los puntos de la trayectoria proveen información de la misma 
calidad. Es por eso que destinaremos un experimento entero a evaluar qué tasa de efectividad obtenemos tomando diferentes subconjuntos 
de puntos de la trayectoria.

\subsection{Bases generales}
Para determinar si cierta combinación de parámetros tuve una mejor performance que otra vamos a basarnos exclusivamente en la tasa de
efectividad de atajadas del arquero. Solamente vamos a considerar si nuestra aproximación converge a la posición final en el experimento
para evaluar el método.
Para generar instancias basadas en una función polinómica de grado $g$ tomamos un punto al azar del campo de juego y otro al azar entre los dos postes 
del arco. Luego tomamos tantos puntos al azar sobre el campo de juego entre el primer punto y la línea de meta hasta alcanzar una cantidad
de puntos igual a $\#p = g + 1$. Con esos puntos generamos el polinomio de Lagrange asociado y con el mismo generamos el resto de los puntos
que queramos para nuestra trayectoria.

\subsection{Experimentos}

\subsubsection{Evaluación del método}
El objetivo de esta experimentación es evaluar si la aproximación de nuestro método realmente converge al valor final del $t_{gol}$ y a 
la posición $y$ por la cual el balón atraviesa la línea de meta. Lo haremos con lo que consideramos disparos básicos de una situación
real de juego. Éstos comprenden disparos con trayectorias que describen rectas, parábolas y curvas generadas por polinomios de tercer
grado. Además introduciremos un ruido de hasta $5\%$ de error, es decir, el valor recibido puede variar hasta en un $5\%$ del valor
original de la función en el punto. Dejaremos de lado a los jugadores y los errores extraordinarios en este experimento ya que nos interesa
simplemente evaluar cómo se comporta el método con trayectorias simples.
Separaremos rectas, parábolas y las curvas de tercer grado. Generaremos instancias aleatorias para cada una pero todas compartirán la misma
cantidad de mediciones. Luego iremos calculando en cada instante la distancia entre el $t_{gol}$ real y nuestra aproximación del mismo y lo
mismo para el lugar por el cual el balón atraviesa la línea de meta. De esta forma podremos ver si nuestra aproximación converge a el
valor final tanto en el tiempo como en la posición.

\subsubsection{Grado del polinomio}
El objetivo de este experimento es analizar si la tasa de efectividad mejora aproximando las funciones $x(t)$ e $y(t)$ con un grado en 
particular de polinomio de cuadrados mínimos. Probaremos primer grado, segundo grado, tercer grado y además compararemos aproximando
con el mayor grado posible. Nuestra hipótesis es que a medida que aumentamos el grado la tasa de efectividad mejorará hasta cierto umbral en 
el cual aumentos en el grado no redundan en mejoras en la aproximación. Esto creemos que se debería a que en general las trayectorias de los 
disparos no describen curvas con más de dos puntos de inflexión.

\subsubsection{Puntos a considerar}
La trayectoria se compone de una serie de puntos, sin embargo, nuestra hipótesis es que los puntos que se corresponden con los instantes más
recientes aportan información de mayor calidad y permiten realizar una aproximación más certera. 


\subsubsection{Errores extraordinarios}


\subsubsection{Jugadores}

\subsubsection{Atajadas con memoria}
