\section{Experimentación}

\subsection{Discusión sobre el método y sus variables}
La efectividad del método estará dada por la calidad de la aproximación del lugar por el cual el disparo llegará a la línea de meta. 
Nuestra aproximación de $x(t)$ nos da el instante en el cual el balón atravesará la línea de meta y luego nuestra aproximación
de $y(t)$ nos dirá el lugar por dónde lo hará. Las variables principales que entran en juego al aproximar una función con 
cuadrados mínimos son las mediciones a considerar, es decir los puntos, y el grado del polinomio con el cual se aproximará la función.
Creemos que un análisis físico sobre las diferentes fuerzas que puedan afectar a la dirección y aceleración del balón puede indicar 
más firmemente qué grados tiene más sentido utilizar para cuadrados mínimos. Sin embargo, el mismo cae fuera del alcance del presente 
trabajo por lo cual nos limitaremos a experimentar con el mayor rango de grados posible y quizás los resultados nos arrojen alguna pista.
Con respecto a qué puntos de la trayectoria utilizar intuimos que no todos los puntos de la trayectoria proveen información de la misma 
calidad. Es por eso que destinaremos un experimento entero a evaluar qué tasa de efectividad obtenemos tomando diferentes subconjuntos 
de puntos de la trayectoria.

\subsection{Bases generales}
Criterios que aplicamos a todos los experimentos:
\begin{itemize}
	\item Para determinar si cierta combinación de parámetros tuvo una mejor performance que otra, vamos a basarnos exclusivamente en la tasa 
	de efectividad de atajadas del arquero. Solamente vamos a considerar si nuestra aproximación converge a la posición final en el experimento para 
	evaluar el método.
	\item Para generar instancias basadas en una función polinómica de grado $g$ tomamos un punto al azar del campo de juego y otro al azar entre los dos postes 
del arco. Luego tomamos tantos puntos al azar sobre el campo de juego entre el primer punto y la línea de meta hasta alcanzar una cantidad
de puntos igual a $\#p = g + 1$. Con esos puntos generamos el polinomio de Lagrange asociado y con el mismo generamos el resto de los puntos
que queramos para nuestra trayectoria.
	\item El grado del polinomio utilizado en los experimentos no est
	\begin{enumerate}
		\item
	\end{enumerate}
\end{itemize}

\subsection{Experimentos}

\subsubsection{Evaluación del método}
El objetivo de esta experimentación es evaluar si la aproximación de nuestro método realmente converge al valor final de 
la posición $y$ por la cual el balón atraviesa la línea de meta. Lo haremos con lo que consideramos disparos básicos de una situación
real de juego. Éstos comprenden disparos con trayectorias de tres tipos: rectas, parábolas y curvas generadas por polinomios de tercer
grado. Separaremos en esos tres tipos y generaremos para cada uno de ellos 100 instancias aleatorias. Todas ellas compartirán 
una cantidad de mediciones igual a 30 y el gráfico que presentaremos se constituirá por los promedios de las instancias de cada tipo. 
Para obtener la información pertinente iremos calculando en cada instante la distancia entre 
el $y_{gol}$ real y nuestra aproximación del mismo. De esta forma podremos ver si nuestra aproximación converge al valor final de 
la posición en la que el balón atraviesa la línea de meta.
Además introduciremos un ruido de hasta $5\%$, es decir, el valor de la medición generada puede variar hasta en un $5\%$ con 
respecto al valor original de la función en el punto. La forma en la que introducimos el ruido es la siguiente:
\begin{enumerate}
	\item cada punto tiene una probabilidad del $50\%$ de llevar ruido o no
	\item el porcentaje de ruido introducido es un valor del intervalo $[0, 5]$ elegido aleatoriamente con una distribución uniforme
	\item el ruido será sumado o restado también con una probabilidad del $50\%$
\end{enumerate}
Dejaremos de lado a los jugadores y los errores extraordinarios en este experimento ya que nos interesa simplemente evaluar cómo se 
comporta el método con trayectorias simples.  

\subsubsection{Grado del polinomio}
El objetivo de este experimento es analizar si la tasa de efectividad mejora aproximando las funciones $x(t)$ e $y(t)$ con un grado en 
particular de polinomio de cuadrados mínimos. Probaremos desde el primer grado hasta el décimo grado. La idea principal detrás de esta 
experimentación es calcular cual es el grado óptimo para que la tasa de efectividad sea lo mejor posible. Se presume que un grado muy 
bajo no va a ser eficiente ya que no puede interpolar muchas puntos y va a ser malo para aproximar curvas algo complicadas, y tampoco 
lo va a ser un grado alto, ya que suelen ser curvas muy oscilantes que pueden introducir error. Para cada grado se testearon todas las instancias 
fijando la cantidad de puntos a considerar (siempre los últimos) y se tomo el promedio de dichos resultados. Se toman instancias de todo tipo, 
tanto de la cátedra como propias.

\subsubsection{Puntos a considerar}
Para esta experimentación se desea ver cómo afecta la cantidad de puntos que se consideran para hallar el polinomio de aproximación de 
cuadrados mínimos. Siempre se van a considerar los últimos puntos medidos. En principio, una cantidad baja de puntos podría no ser 
buena ya que no son suficientes para estimar curvas complicadas, tampoco sería bueno que la cantidad sea demasiado alta porque la 
estimación podría ser más propensa a errores, especialmente considerando la presencia de jugadores que pueden alterar el curso de la 
pelota de manera inesperada y que no corresponda con alguna curva "natural''. Como en la experimentación anterior, se usan todo tipo 
de instancias.


%\subsubsection{Errores extraordinarios}
