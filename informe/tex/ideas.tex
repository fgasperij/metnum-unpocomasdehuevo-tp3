X(t) nos va a decir cuándo e Y(t) nos va a decir dónde. Por lo tanto, la información que extraemos de X(t) es más que nada 
la velocidad a la que se aproxima a la línea de meta. Conociendo esa velocidad podemos saber cuánto tiempo va a tener Y(t) para laburar
y llevarla hasta el lugar que pueda. 

Cómo estamos modelando un disparo de fútbol no tiene sentido considerar una aproximación lineal a X(t) ya que la velocidad siempre va 
a tener alguna aceleración. Entonces tenemos que X(t) tiene una velocidad y una aceleración. La aceleración probablemente vaya decreciendo
más y más y también la velocidad después de cierto umbral (cuando la aceleración deja de ser positiva). A menos que, por ejemplo un 
jugador le aplique una fuerza. Con Y(t) todo esto aplica también.


OUTLIERS
en lugar de ignorarlos deberíamos extrapolarlos para no sumarle velocidad al pedo a la funcións