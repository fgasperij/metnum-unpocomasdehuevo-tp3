\section{Introducción}

En el presente trabajo nos proponemos estudiar el problema de atajar un disparo de fútbol hacia la portería. Puntualmente tendremos control
sobre el arquero y deberemos decidir en cada instante de tiempo en qué dirección debe moverse para lograr atajar el disparo. 
Impondremos un límite de desplazamiento al arquero por instante de tiempo. Es decir, cada vez que decidamos moverlo podremos hacerlo
como máximo en una magnitud menor o igual al límite máximo de desplazamiento impuesto de antemano.
La información con la que contaremos para decidir los movimientos del arquero son mediciones sobre la posición de la pelota para determinados
instantes cuyo orden conoceremos. Las mediciones son realizadas a intervalos regulares de tiempo, de esta forma, nos queda discretizado el mismo en los intantes 
recibidos y sólo podemos saber el orden de los mismos pero ninguna información sobre la pelota entre dos instantes dados. Además, sabemos
que las mediciones pueden tener ruido o errores propias de la misma medición.
Los postes del arco están ubicados en coordenadas fijas, y la línea de gol es el segmento entre estos dos puntos.
Se marca un gol cuando la pelota cruza este segmento. Visto desde arriba, la pelota es un círculo de radio
determinado y el arquero se representa mediando un segmento paralelo a la línea de gol ubicado sobre la misma.
Inicialmente el arquero se encuentra en un punto entre estos dos segmentos y en cada paso se le indica al arquero
en qué sentido y en qué magnitud debe moverse, ya que la dirección sobre la que se mueve es siempre la misma, la 
línea de gol.
Además contaremos con las posiciones de los jugadores rivales en el campo de juego. Para simplificar el alcance y
el modelo propuesto asumiremos que la posición de los jugadores no varía con el tiempo aunque sí pueden intervenir
en el juego y desviar la pelota con un remate al arco o un pase a otro jugador.
En cada instante nuestro desafío reside en decidir cómo utilizar la información de los instantes anteriores y el actual para aproximar 
la posición por la que la pelota ingresará al arco y así mover al arquero a esa posición para atajarla. Para la predicción de la 
posición por donde entrará la pelota, nos basamos en el método de \emph{cuadrados mínimos lineales}. 
Este método es particularmente adecuado para los casos en que se introducen errores de medición\cite[\emph{3.2}]{heath} como ocurre 
en las mediciones de la trayectoria del balón.
\par
La trayectoria de la pelota la formalizaremos mediante una función $p:\mathbb{R} \to \mathbb{R}^2$, $p(t) = (x(t),y(t))$ donde 
$t$ es el tiempo y $x(t)$ la coordenada $x$ del balón en el instante $t$ e $y(t)$ la coordenada $y$ del balón en el mismo instante considerando
que abstraemos al campo de juego como puntos en un eje de coordenadas en el cual los arcos se ubican paralelos al eje $y$.
El método de cuadrados mínimos\cite{burden}  lo utilizaremos para calcular los coeficientes de dos polinomios $P_x(t)$ y $P_y(t)$ que aproximen a las funciones 
$x(t)$ e $y(t)$, respectivamente, de forma tal que el error cuadrático sea mínimo:
\begin{displaymath}
  E_c(P_{x/y}) = \frac{\sum_{k=1}^{n}({x/y}_k-P_{x/y}(t_k))^2}{n} 
\end{displaymath}
con $n$ igual a la cantidad de mediciones y $x/y_k$ el valor de la coordenada $x$ o $y$ de la medición realizada en el instante $k$.

